\documentclass{book}


% ==================================================================================================================================
% HEADER
% Documents packages, environements, page configuration

\usepackage[french]{babel}
\usepackage[T1]{fontenc}
\usepackage{ragged2e}
\usepackage{amsfonts}
\usepackage{systeme}
\usepackage{amsmath}
\usepackage{amssymb}
\usepackage{comment}
\usepackage{multicol}
\usepackage{lipsum} 
\usepackage{graphicx}
\usepackage{stmaryrd}
\usepackage{wrapfig}
\usepackage{colortbl}
\usepackage{cellspace}
\usepackage{ntheorem}
\usepackage{lmodern}
\usepackage{mathtools}
\usepackage{ragged2e}
\usepackage{tabularx}
\usepackage{titlepic}
\usepackage{fancyhdr}
\usepackage{caption}
\usepackage{xcolor}
\usepackage[linkbordercolor=white]{hyperref}
\usepackage[T1]{fontenc}
\usepackage{lmodern}
\usepackage{listings}
\usepackage{tikz}
\usepackage{tkz-graph}
\usepackage{pgfplots}
\usepackage{mdframed}
\usepackage{xparse}
\usepackage{enumitem}
\usepackage{titlesec}
\usepackage[top=3cm, bottom=3cm, left=3.5cm, right=3.5cm]{geometry}
\usepackage{etoolbox} % pour tester la présence d'un argument optionnel

% \pagestyle{empty}

\usetikzlibrary{cd}
\usetikzlibrary{patterns}
\usetikzlibrary{angles,quotes}
\usetikzlibrary{intersections, 3d, calc}

\usepgfplotslibrary{fillbetween}


% - - - - - - - - - - - - - - - - - - - - - - - - - 
%                   PAGE SETTINGS
% - - - - - - - - - - - - - - - - - - - - - - - - - 

\setlength{\columnseprule}{1pt}               % ligne de séparation entre les colonnes
\def\columnseprulecolor{\color{black}}

\newcolumntype{C}{>{$\displaystyle}Sc<$}      % style d'affichage et taille 
\cellspacetoplimit=5pt                        % de la séparation entre colonnes
\cellspacebottomlimit=5pt

\renewcommand{\thechapter}{\arabic{chapter}}  % Set chapters numbers to 0

% Redéfinition de \part avec image optionnelle
\renewcommand{\part}[2][]{%
  \clearpage
  \refstepcounter{part} % incrémente le compteur de partie
  \addcontentsline{toc}{part}{\thepart\space #2} % ajoute au sommaire

  \begin{center}
    {\Huge\bfseries #2\par}
    \vspace{4em}  % Augmente l'écart à 2em
    \ifstrempty{#1}{}{%
      \includegraphics[width=0.4\textwidth]{#1}\par
    }
  \end{center}
  \vspace{2em}
}



% - - - - - - - - - - - - - - - - - - - - - - - - - 
%                 MATHS SHORTHANDS
% - - - - - - - - - - - - - - - - - - - - - - - - - 

\newcommand{\C}{\mathbb{C}}
\newcommand{\R}{\mathbb{R}}
\newcommand{\Q}{\mathbb{Q}}
\newcommand{\Z}{\mathbb{Z}}
\newcommand{\N}{\mathbb{N}}
\newcommand{\U}{\mathbb{U}}
\newcommand{\K}{\mathbb{K}}
\newcommand{\E}{\mathbb{E}}
\newcommand{\V}{\mathbb{V}}
\newcommand{\Lc}{\mathbb{L}}
\newcommand{\Fc}{\mathbb{F}}

\newcommand{\M}{\mathcal{M}}
\newcommand{\B}{\mathcal{B}}
\newcommand{\F}{\mathcal{F}}

%\renewcommand{\epsilon}{\varepsilon}
%\renewcommand{\phi}{\varphi}
%\renewcommand{\rho}{\varrho}

\renewcommand{\labelitemi}{\textbullet} % Utiliser des points noirs (•)

\newcommand{\myP}{\mathbb{P}}


% - - - - - - - - - - - - - - - - - - - - - - - - - 
%                 MODS NOTATION
% - - - - - - - - - - - - - - - - - - - - - - - - - 

\theorembodyfont{\upshape}

% Définir un environnement pour encadrer les définitions avec un titre
\NewDocumentEnvironment{definition}{O{}}
{
  \begin{mdframed}[linewidth=0pt,linecolor=gray,backgroundcolor=gray!10,roundcorner=5pt]
  \textbf{Définition}%
  \IfNoValueTF{#1}{}{~(\textbf{#1})} % Affiche le titre entre parenthèses et en gras s'il est fourni
  . % Point à la fin
}
{
  \end{mdframed}
}

% Définir un environnement pour encadrer les théorèmes avec un titre
\NewDocumentEnvironment{theorem}{O{}}
{
  \samepage
  \begin{mdframed}[linewidth=1pt,linecolor=darkgray,backgroundcolor=darkgray!10,roundcorner=5pt]
  \textbf{Théorème}%
  \IfNoValueTF{#1}{}{~(\textbf{#1})} % Affiche le titre entre parenthèses et en gras s'il est fourni
  . % Point à la fin
}
{
  \end{mdframed}
}

% Définir un environnement pour encadrer les corollaires
\NewDocumentEnvironment{corollary}{O{}}
{
  \begin{mdframed}[linewidth=1pt,linecolor=gray,backgroundcolor=gray!10,roundcorner=5pt]
  \textbf{Corollaire}%
  \IfNoValueTF{#1}{}{~(\textbf{#1})} % Affiche le titre entre parenthèses et en gras s'il est fourni
  . % Point à la fin
}
{
  \end{mdframed}
}


% Définir un environnement pour encadrer les critères
\NewDocumentEnvironment{criteria}{O{}}
  {
    \begin{mdframed}[linewidth=1pt,
                      linecolor=gray,
                      backgroundcolor=gray!10,
                      roundcorner=5pt
                      ]
    \textbf{Critère}%
    \IfNoValueTF{#1}{}{~(\textbf{#1})} % Affiche le titre entre parenthèses et en gras s'il est fourni
    . % Point à la fin
  }
  {
    \end{mdframed}
  }

% Définir un environnement pour encadrer les prorpiétés avec un titre
\NewDocumentEnvironment{prop}{O{}}
{
  \begin{mdframed}[linewidth=0pt,linecolor=gray,backgroundcolor=gray!10,roundcorner=5pt]
  \textbf{Propriété}%
  \IfNoValueTF{#1}{}{~(\textbf{#1})} % Affiche le titre entre parenthèses et en gras s'il est fourni
  . % Point à la fin
}
{
  \end{mdframed}
}
				

\theoremstyle{plain}
\newtheorem*{remark}{Remarque}
\newtheorem*{proposition}{Proposition}
\newtheorem*{lemma}{Lemme}
%\newtheorem*{prop}{Propriété}
\newtheorem*{proof}{Démonstration}
\newtheorem*{example}{Exemple}

% Créer un environnement qui empêche les coupures de page
\newenvironment{nobreakproposition}
{
  \par\begingroup\samepage % Empêcher les coupures de page
  \begin{proposition}
}
{
  \end{proposition}
  \par\endgroup
}


% - - - - - - - - - - - - - - - - - - - - - - - - - 
%             Itemize env settings
% - - - - - - - - - - - - - - - - - - - - - - - - - 

\setlist[itemize,1]{label=\textbullet}
\setlist[itemize,2]{label=\textbullet}
\setlist[itemize,3]{label=\textbullet}
\setlist[itemize,4]{label=\textbullet}


% - - - - - - - - - - - - - - - - - - - - - - - - - 
%             Tableofcontents settings
% - - - - - - - - - - - - - - - - - - - - - - - - - 


% % Profondeur des mini-tables des matières
% \setcounter{minitocdepth}{2}  % Jusqu’aux sous-sections
% \setcounter{parttocdepth}{1}  % Jusqu’aux chapitres dans les parties

% % Configurer l’apparence
% \mtcsetrules{minitoc}{off}  % Désactiver les lignes pour les chapitres
% \mtcsetrules{parttoc}{off}   % Désactiver les lignes pour les parties

% % Si tu veux que la table principale n'affiche que les parties :
% \setcounter{tocdepth}{-1}  % N'afficher que les \part dans la table principale


% - - - - - - - - - - - - - - - - - - - - - - - - - 
%                       TITLE
% - - - - - - - - - - - - - - - - - - - - - - - - - 


% ==================================================================================================================================
% Setting Title 

% Titre du document
\title{Révisions Master IMA}
\author{Axel PIGEON}
\date{\today}

\setlength{\parindent}{0pt}
\renewcommand{\labelitemi}{\textbullet} % Utiliser des points noirs (•)

\begin{document}


% ==================================================================================================================================
% CONTENT


% Page de titre 
\maketitle

% Table des matières 
\tableofcontents
\newpage

\setlength{\parindent}{0pt}

% Inclusion des chapitres 
\part{Probabilités}
\chapter{Espaces Probabilisés et Variables Aléatoires}

\justify

\setlength{\parindent}{0pt}
\renewcommand{\labelitemi}{\textbullet} % Utiliser des points noirs (•)

% ==================================================================================================================================
% Introduction 


% ==================================================================================================================================
% Espaces Probabilisés et Mesures 

\section{Espaces Probabilisés, Mesures et Variables Aléatoires}

\subsection{Univers}

Introduisons les concepts fondamentaux des probabilités, les univers et les espaces probabilisés. 

\begin{definition}[Univers]
    On appelle univers $\Omega$ pour une expérience aléatoire, l'ensemble de toutes les issues (situations finales) possibles 
    de cette expérience aléatoire. Chaque élément $ \omega \in \Omega$ représente une \textbf{issue} de cette expérience aléatoire.  
\end{definition}

\begin{example}
    \begin{itemize}
        \item Pour une expérience aléatoire de lancer de dé, il existe 6 issues possibles correspondant aux 6 faces du dé. 
        On a donc $\Omega = \{1, 2, 3, 4, 5, 6\}$. 
        \item Si on pioche un boule dans une urne contenant une boule rouge et deux boules noires, on a 
        $ \Omega = \{\text{rouge}, \text{noir}\}$. 
    \end{itemize}
\end{example}

A partir d'un univers, on peut définir la notion d'espace probabilisé. Plus complexe, la définition nécessaire les prérequis 
du cours d'intégration et de théorie de la mesure. 

\begin{definition}[Espace Probabilisé]
    Un espace probabilisé est un \textbf{triplet} $ (\Omega, \mathcal{F}, \myP)$ où :
    \begin{itemize}
        \item $\Omega$ est un univers. 
        \item $ \mathcal{F}$ est une tribu ($\sigma$-algèbre) sur $ \Omega$. 
        \item $ \myP$ est une mesure de probabilité sur $ \mathcal{F}$ (voir plus loin). 
    \end{itemize}
\end{definition}



\subsection{Évènements, Issues et Mesure de Probabilité }

\begin{definition}[Évènement]
    Soit $\Omega$ un univers. On définit un évènement de $\Omega$ comme un sous-ensemble $A \subseteq \Omega$. 
\end{definition}

\begin{remark}
    Comme définit au début, les \textbf{issues} $ \omega \in \Omega$ correspondent à des résultats élémentaires de l'expérience 
    aléatoire, à ne pas confondre avec les évènements. 
    Dans notre expérience de lancer de dé, $\{2\} \in \Omega$ est \textbf{l'issue} correspondant à "obtenir un 2" et 
    $A = \{1,2\} \subset \Omega$ est \textbf{l'évènement} correspondant à "le résultat est inférieur ou égal à 2". 
\end{remark}

Par construction, $ \mathcal{F}$ contient donc tous les évènements et issues possibles de l'expérience aléatoire. 
Elle est dont "plus complète" que $\Omega$, on retrouve les propriétés des espaces mesurables, vus en intégration en Licence. 

\begin{definition}[Mesure de Probabilité]
    Soit un espace probabilisé $(\Omega, \mathcal{F}, \myP)$. Une mesure de probabilité $ \myP$ 
    sur $ \mathcal{F}$ est une mesure (au sens de la théorie de la mesure) qui vérifie : 
    \begin{enumerate}
        \item $ \myP : \mathcal{F} \longrightarrow [0,1]$
        \item $ \myP(\Omega) = 1$ 
    \end{enumerate}
\end{definition}

\begin{remark}[Rappel : Mesure]
    Une fonction $ \mu : (X, \mathcal{B}) \longrightarrow \overline{\R_+}$ telle que : 
    \begin{enumerate}
        \item $ \mu(\emptyset) = 0 $ 
        \item \textbf{(Sigma-additivité)} : $ \forall (A_n)_{n \in \N}$ suite de parties mesures \textbf{deux à deux disjointes}, on ait : 
            \[ \mu \left( \bigcup_{n \in \N} A_n \right) = \sum_{n \in \N} \mu(A_n) \]
        est appelée mesure sur l'espace $(X, \mathcal{B}, \mu)$ alors appelé \textbf{espace mesuré}. 
    \end{enumerate}
\end{remark}

\section{Variable Aléatoire}

\subsection{Définition et Loi}

Pour pouvoir quantifier des calculs de probabilités ou ce que nous appellerons plus tard des lois, nous devons définir 
les variables aléatoires. 

\begin{definition}[Variable Aléatoire]
    Une variable aléatoire est une fonction mesurable qui associe une valeur numérique à chaque issue d'un espace probabilisé. 

    \vspace{0.5cm}

    Plus formellement, une variable aléatoire $X$ sur un espace probabilisé $(\Omega, \mathcal{F}, \myP)$ 
    est une fonction $ X : \Omega \longrightarrow \R$ telle que, pour tout ensemble $B \in \mathcal{B}_\R$ (tribu borélienne), 
    on ait $X^{-1} (B) \subset \mathcal{F}$. 

    \vspace{0.5cm}

    On définit l'ensemble des valeurs possibles de la variable aléatoire comme l'image de $\Omega$ par $X$ noté $X(\Omega)$. 
\end{definition}

La mesurabilité d'une variable aléatoire permet donc garantir que les évènements associés aux valeurs de la variable 
aléatoire sont bien mesurables par la mesure de probabilité. 

\vspace{0.5cm}

Maintenant que nous avons définit formellement le concept de variable aléatoire, on peut lier cette définition 
à celle des évènements. En effet, une variable aléatoire est une fonction mesurable sur un espace probabilisé $(\Omega, \mathcal{F}, \myP)$
telle que : 
    \[ X : \Omega \longrightarrow \R \quad \forall B \in \mathcal{B}_\R, X^{-1} (B) \subset \mathcal{F} \] 
On peut alors caractériser un évènement $A$ comme la préimage d'un sous-ensemble de $\R$ par $X$ de la façon suivante. 
    \[ A = X^{-1}(B) \; \text{pour un certain} \; B \in \mathcal{B}_\R \] 
D'où la définition suivante...

\begin{definition}[Evènement d'une expérience aléatoire]
    Soit $X$ une variable aléatoire réelle définie sur un univers $\Omega$. 
    On appelle évènement $[X = x]$ de l'expérience aléatoire l'ensemble des issues possibles correspondant à cet évènement $x \in \R$ . 
    Autrement dit :
        \[ [X = x] = \{w \in \Omega \; | \; X(\omega) = x\} = X^{-1}(x) \] 
\end{definition}

\begin{definition}[Loi]
    Soit $X$ une variable aléatoire. On appelle loi de $X$ la donnée de toutes les probabilités $ \myP(X = x)$ pour 
    tout $x \in X(\Omega)$.  
\end{definition}

Pour donner la loi d'une variable aléatoire, il faut d'abord déterminer le support de la variable aléatoire puis en suite 
calculer la probabilité de chaque issue. 
On note le résultat dans un tableau pour plus de praticité. 

\begin{example}
    Soit l'exprérience aléatoire du lancer d'un dé à 6 faces non truqué. On a :
        \[ \Omega = \{1,2,3,4,5,6\} \]
    Nous nous trouvons dans une situation d'équiprobabilité d'où :
        \[ \forall x \in X(\Omega), \quad \myP(X = x) = \frac{1}{6} \]
    D'où le tableau suivant :
    \begin{center}
        \begin{tabular}{c|c|c|c|c|c|c}
            $\Omega$ & 1 & 2 & 3 & 4 & 5 & 6 \\
            \hline 
            $ \myP(X = x)$ & $\frac{1}{6}$ & $\frac{1}{6}$ & $\frac{1}{6}$ & $\frac{1}{6}$ & $\frac{1}{6}$ & $\frac{1}{6}$ \\ 
        \end{tabular}
    \end{center}  
\end{example}

\subsection{Variables Discrètes et Continues}

Selon la nature de l'espérience aléatoire et de l'univers choisis, on distingue deux grands types de variables aléatoires : 
les variables aléatoires \textbf{discrètes} et \textbf{continues}. Cette distinction est fondamentale car elle 
caractérise la façon dont on exprime et calcule ensuite les probabilités.

\begin{definition}[Variable aléatoire discrète]
    Une variable aléatoire $X$ définie sur un espace probabilisé $(\Omega, \mathcal{F}, \myP)$ est dite 
    \textbf{discrète} si l'ensemble de ses valeurs possibles $X(\Omega)$ est un ensemble \textbf{fini ou dénombrable}. 
    Dans ce cas, la loi de probabilité de $X$ est donnée par une fonction de masse et les probabilités 
    s'expriment comme des sommes. 
\end{definition}

\begin{example}
    Le résultat d'un lancer de dé est une variable aléatoire discrète. 
    Par exemple, si $X$ désigne le résultat d’un lancer de dé, alors $X(\Omega) = \{1,2,3,4,5,6\}$.
\end{example}

\begin{definition}[Variable aléatoire continue]
    Une variable aléatoire $X$ est dite \textbf{continue} si elle peut prendre une infinité non dénombrable de valeurs, typiquement un intervalle de $\R$. 
    Dans ce cas, il n'existe pas de fonction de masse mais une \textbf{densité de probabilité}, et les probabilités s'expriment 
    par des \textbf{intégrales}.
\end{definition}


\begin{example}
    Le temps d'attente avant un événement (modélisé par une loi exponentielle), ou la taille d'une personne (loi normale), 
    sont des variables continues. Si $X$ est la taille d’un individu, alors $X(\Omega) \subseteq \R$ est un intervalle de réels.
\end{example}

\begin{remark}
    La distinction entre lois discrètes et lois continues repose donc sur la nature de la mesure de probabilité utilisée :
    \begin{itemize}
        \item \textbf{Mesure de comptage} (ou somme de Dirac) $\Rightarrow$ lois discrètes.
        \item \textbf{Mesure de Lebesgue} (avec densité) $\Rightarrow$ lois continues.
    \end{itemize}
\end{remark}

Nous allons maintenant étudier les deux grands types de lois de probabilité selon la nature de la variable aléatoire : 

\begin{itemize}
    \item Dans le cas discret : lois binomiale, géométrique, de Poisson...
    \item Dans le cas continu : loi uniforme, loi exponentielle, loi normale...
\end{itemize}

\chapter{Variables Aléatoires Réelles Discrètes}

\justify

\setlength{\parindent}{0pt}
\renewcommand{\labelitemi}{\textbullet} % Utiliser des points noirs (•)

% ==================================================================================================================================
% Introduction 


% ==================================================================================================================================
% Variables Aléatoires Réelles Discrètes


\section{Espérance, Variance et Écart-type}

\begin{definition}[Espérance]
    Soit $X : \Omega \longrightarrow \R$ une variable aléatoire discrète. 
    On appelle espérance l'application $ \E : X(\Omega) \longrightarrow \R $ qui calcule la moyenne de $X$ pondérée par les valeurs qu'elle prend. 
    Plus formellement :
        \[ \boxed { \E(X) = \sum_{x \in X(\Omega)} x \myP(X = x) }\] 
\end{definition}

\begin{prop}[Espérance]
    L'espérance est une fonction linéaire. Autrement dit, pour toutes variables aléatoires $X,Y$ sur  un univers $\Omega$, 
    et pour tout $a,b \in \R$, on a :
        \[ \E(X+Y) = \E(X) + \E(Y) \quad \E(aX + b) = a\E(X) + b \]
\end{prop}

Lors d'une expérience aléatoire, par exemple un jeu d'argent l'espérance représente le gain moyen d'un joueur par partie 
s'il joue un grand nombre de fois. Son signe permet de savoir si le jeu est dit équitable (autant de chances de gagner que 
de perdre). 

\vspace{0.3cm}

Il peut souvent arriver que l'on veuille appliquer une fonction à notre variable aléatoire. 
Un théorème nous permet alors simplement de calculer l'espérance de cette "nouvelle" variable aléatoire. 

\begin{theorem}[Transfert]
    Soit $X$ une variable aléatoire discrète sur un univers $\Omega$ et $g : \R \longrightarrow \R$ une application. 
    L'espérance de la variable aléatoire $g(X) : \Omega \longrightarrow \R$ est l'application $ \E(g(X)) : \mathcal{F}(\Omega) \longrightarrow \R$ 
    telle que : 
        \[ \boxed{ \E(g(X)) = \sum_{x \in X(\Omega)} g(x) \myP(X = x) } \]  
\end{theorem}

\begin{definition}[Variance et écart-type]
    Soit $X$ une variable aléatoire discrète sur un univers $\Omega$. On appelle variance l'application 
    $\V : X(\Omega) \longrightarrow \R$ telle que :
        \[ \boxed { \V(X) = \sum_{x \in X(\Omega)} (x - \E(X))^2 \myP(X = x) } \]
    De même, on appelle écart type l'application $ \sigma : X(\Omega) \longrightarrow \R$ telle que : 
        \[ \boxed{ \sigma(X) = \sqrt{\V(X)} } \]  
\end{definition}

La variance permet de mesurer la dispersion de la variable aléatoire autour de son espérance. 

\begin{remark}
    Les notions d'espérance, variance et écart-type sont définies par des sommes potentiellement infinies. 
    Il se peut donc que dans le cas de variables aléatoires définies sur des univers infinis, leur espérance, 
    variance et écart-type n'existent pas. Une étude de convergence de la somme est donc judicieuse. 
    En revanche pour les variables aléatoires définies sur un univers fini ces valeurs existent bien. 
\end{remark}

\begin{theorem}[Formule de König-Huygens]
    Soit $X$ une variable aléatoire sur un univers $\Omega$ fini. On a alors : 
        \[ \boxed{ \V(X) = \E(X^2) - \E(X)^2 } \]
\end{theorem}

\begin{remark}
    Le calcul de la variance d'une variable aléatoire réelle finie est donc assez facile quand on le met en relation 
    avec la formule de König-Huygens et la formule du transfert...
\end{remark}

A partir de toutes ces formules, on peut en déduire quelques propriétés sympatiques sur la variance :

\begin{prop}[Variance]
    Soit $X$ une variable aléatoire finie et $a,b \in \R$, on a :
        \[ \V(aX+b) = a^2 \V(X) \quad \V(X + b) = \V(X) \]
    La variance est donc assez similaire à une forme quadratique et est invariante par translation. 
    De plus, la variance d'une variable aléatoire est invariable par translation des valeurs de la variable aléatoire. 
\end{prop}

\begin{theorem}[Inégalité de Markov]
    Si $X$ est une variable aléatoire réelle discrète \textbf{positive} ou nulle sur $\Omega$ d'espérance $\E(X)$, alors :
        \[ \forall a \in ]0, + \infty [ \quad P(X \geqslant a) \leqslant \dfrac{\E(X)}{a} \]
    Cela fournit un majorant de la probabilité que $X$ dépasse un seuil $a$ donné.
    Ce résultat est particulièrement utile pour borner des queues de distributions sans connaître leur loi exacte.
\end{theorem}


\section{Principales Lois}

Abordons en détail maintenant quelques lois usuelles à connaître sur le bout des doigts. 
Ces lois permettent de modéliser la plupart des expériences aléatoires. 

\subsection{Loi Uniforme}

\begin{definition}[Loi Uniforme]
    Soit $n \in \N^*$. On dit qu'un variable aléatoire $X$ suit une \textbf{loi uniforme} sur $ \llbracket 1, n \rrbracket $ 
    lorsque son support est $X(\Omega) = \llbracket 1, n \rrbracket $ et chaque issue a la même probabilité de se produire. 
    Autrement dit : 
        \[ \forall x \in \llbracket 1, n \rrbracket, \quad P(X = x) = \frac{1}{n} \]
    On note alors $X \sim \mathcal{U}(\llbracket 1, n \rrbracket)$. 
\end{definition}

\begin{proposition}
    Soit $X \sim \mathcal{U}(\llbracket 1, n \rrbracket)$ alors l'espérance et la variance de $X$ sont de la forme : 
        \[ \E(X) = \frac{n+1}{2} \quad \text{et} \quad \V(X) = \frac{n^2-1}{12}  \] 
\end{proposition}

\subsection{Loi de Bernoulli}

\begin{definition}[Loi de Bernoulli]
    Une variable aléatoire $X$ suit une \textbf{loi de Bernoulli} de paramètre $p \in ]0,1[$ si il n'existe que deux issues 
    possibles $X(\Omega) = \{ 0,1 \}$ telles que : 
        \[ \myP(X = 1) = p \quad \text{et} \quad \myP(X = 0) = 1 - p \] 
    On note alors $X \sim \mathcal{B}(p)$.  
\end{definition}

\begin{proposition}
    Soit $X \sim \mathcal{B}(p)$ alors l'espérance et la variance de $X$ sont de la forme :
        \[ \E(X) = p \quad \text{et} \quad \V(X) = p(1-p)  \] 
\end{proposition}

\subsection{Loi Binomiale}

L'expérience aléatoire consistant à répéter $n \in \N$ fois une expérience de Bernoulli de paramètre $ p \in ]0,1[$
de {manière indépendante} est appelée \textbf{schéma de Bernoulli} de paramètres $n$ et $p$. 

\begin{definition}[Loi Binomiale]
    La variable aléatoire $X$ égale au \textbf{nombre de succès} d'un schéma de Bernoulli suit une {loi binomiale} 
    de paramètres $n$ et $p$. 

    On note alors $X \sim \mathcal{B}(n,p)$. 
\end{definition}

\begin{proposition}
    Soit $X \sim \mathcal{B}(n,p)$. On a alors $X(\Omega) = \llbracket 0, n \rrbracket$ et pour tout $k \in \N$ tel que 
    $0 \leqslant k \leqslant n $, {la probabilité d'obtenir $k$ succès} est donnée par :
        \[ \myP(X = k) = \binom{n}{k} \times p^k \times (1-p)^{n-k}  \] 
\end{proposition}

\begin{proposition}
    Soit $X \sim \mathcal{B}(n,p)$ alors l'espérance et la variance de $X$ sont de la forme :
        \[ \E(X) = np \quad \text{et} \quad \V(X) = np(1-p)  \] 
\end{proposition}

\subsection{Loi géométrique}

\begin{definition}[Loi géométrique]
    Une variable aléatoire $X$ suit une loi géométrique de paramètre $p \in ]0,1[$ lorsque $X(\Omega) = \N^*$ et que 
        \[ \forall k \in \N^*, \quad \myP(X = k) = p(1-p)^{k-1}  \] 
    On note alors $X \sim \mathcal{G}(p)$. 
\end{definition}

Une loi géométrique représente le temps d'attendre du premier succès d'un espérience de Bernoulli. 
Autrement dit X est le rang de l'épreuve ayant mené au premier succès. 

\begin{proposition}
    Soit $X \sim \mathcal{G}(p)$ où $p \in ]0,1[$ l'espérance et la variance de $X$ sont de la forme :
        \[ \E(X) = \frac{1}{p} \quad \text{et} \quad \V(X) = \frac{1-p}{p^2} \] 
\end{proposition}

\subsection{Loi de Poisson}

\begin{definition}[Loi de Poisson]
    Une variable aléatoire $X$ suit une loi de Poisson de paramètre $\lambda > 0$ lorsque $X(\Omega) = \N$ et 
        \[ \forall k \in \N, \quad P(X = k) = e^{- \lambda} \times \frac{\lambda^k}{k!} \]  
    On note alors $ X \sim \mathcal{P}(\lambda)$. 
\end{definition}

Une loi de Poisson modélise le nombre d'événements se produisant dans un intervalle de temps ou d’espace donné, lorsque ces événements sont rares et indépendants.
Par exemple, elle peut modéliser le nombre de voitures passant par un péage en une journée, ou le nombre de fautes de frappe sur une page de texte.

\begin{proposition}
    Soit $ X \sim \mathcal{P}(\lambda)$ alors l'espérance et la variance de $X$ sont de la forme :
        \[ \E(X) = \lambda \quad \text{et} \quad \V(X) = \lambda  \] 
\end{proposition}

\subsection*{Résumé des lois discrètes usuelles}

\begin{center}
    \begin{tabular}{|c|c|c|c|}
        \hline
        \textbf{Loi} & \textbf{Support} & \textbf{Espérance $\E(X)$} & \textbf{Variance $\V(X)$} \\
        \hline
        $\mathcal{U}(\llbracket 1, n \rrbracket)$ & $\llbracket 1, n \rrbracket$ & $\frac{n+1}{2}$ & $\frac{n^2 - 1}{12}$ \\
        \hline
        $\mathcal{B}(p)$ & $\{0,1\}$ & $p$ & $p(1-p)$ \\
        \hline
        $\mathcal{B}(n, p)$ & $\llbracket 0, n \rrbracket$ & $np$ & $np(1-p)$ \\
        \hline
        $\mathcal{G}(p)$ & $\N^*$ & $\frac{1}{p}$ & $\frac{1-p}{p^2}$ \\
        \hline
        $\mathcal{P}(\lambda)$ & $\N$ & $\lambda$ & $\lambda$ \\
        \hline
    \end{tabular}
\end{center}


Les lois étudiées jusque-là sont toutes des lois discrètes, c’est-à-dire que la variable aléatoire prend un nombre fini ou dénombrable de valeurs.  
Dans de nombreux cas, on souhaite modéliser des phénomènes à valeurs continues (comme une durée, une taille, une température, etc.).  
Nous allons maintenant introduire les lois continues les plus classiques : loi uniforme, loi exponentielle, loi normale, etc.




\end{document}