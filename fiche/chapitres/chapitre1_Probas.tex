\chapter{Probabilités}

\justify

\setlength{\parindent}{0pt}
\renewcommand{\labelitemi}{\textbullet} % Utiliser des points noirs (•)

% ==================================================================================================================================
% Introduction 


% ==================================================================================================================================
% Espaces Probabilisés et Mesures 

\section{Espaces Probabilisés et Mesures}

\subsection{Univers}

Introduisons les concepts fondamentaux des probabilités, les univers et les espaces probabilisés. 

\begin{definition}[Univers]
    On appelle univers $\Omega$ pour une expérience aléatoire, l'ensemble de toutes les issues (situations finales) possibles 
    de cette expérience aléatoire. Chaque élément $ \omega \in \Omega$ représente une \textbf{issue} de cette expérience aléatoire.  
\end{definition}

\begin{example}
    \begin{itemize}
        \item Pour une expérience aléatoire de lancer de dé, il existe 6 issues possibles correspondant aux 6 faces du dé. 
        On a donc $\Omega = \{1, 2, 3, 4, 5, 6\}$. 
        \item Si on pioche un boule dans une urne contenant une boule rouge et deux boules noires, on a 
        $ \Omega = \{\text{rouge}, \text{noir}\}$. 
    \end{itemize}
\end{example}

A partir d'un univers, on peut définir la notion d'espace probabilisé. Plus complexe, la définition nécessaire les prérequis 
du cours d'intégration et de théorie de la mesure. 

\begin{definition}[Espace Probabilisé]
    Un espace probabilisé est un \textbf{triplet} $ (\Omega, \mathcal{F}, \myP)$ où :
    \begin{itemize}
        \item $\Omega$ est un univers. 
        \item $ \mathcal{F}$ est une tribu ($\sigma$-algèbre) sur $ \Omega$. 
        \item $ \myP$ est une mesure de probabilité sur $ \mathcal{F}$ (voir plus loin). 
    \end{itemize}
\end{definition}



\subsection{Évènements, Issues et Mesure de Probabilité }

\begin{definition}[Évènement]
    Soit $\Omega$ un univers. On définit un évènement de $\Omega$ comme un sous-ensemble $A \subseteq \Omega$. 
\end{definition}

\begin{remark}
    Comme définit au début, les \textbf{issues} $ \omega \in \Omega$ correspondent à des résultats élémentaires de l'expérience 
    aléatoire, à ne pas confondre avec les évènements. 
    Dans notre expérience de lancer de dé, $\{2\} \in \Omega$ est \textbf{l'issue} correspondant à "obtenir un 2" et 
    $A = \{1,2\} \subset \Omega$ est \textbf{l'évènement} correspondant à "le résultat est inférieur ou égal à 2". 
\end{remark}

Par construction, $ \mathcal{F}$ contient donc tous les évènements et issues possibles de l'expérience aléatoire. 
Elle est dont "plus complète" que $\Omega$, on retrouve les propriétés des espaces mesurables, vus en intégration en Licence. 

\begin{definition}[Mesure de Probabilité]
    Soit un espace probabilisé $(\Omega, \mathcal{F}, \myP)$. Une mesure de probabilité $ \myP$ 
    sur $ \mathcal{F}$ est une mesure (au sens de la théorie de la mesure) qui vérifie : 
    \begin{enumerate}
        \item $ \myP : \mathcal{F} \longrightarrow [0,1]$
        \item $ \myP(\Omega) = 1$ 
    \end{enumerate}
\end{definition}

\begin{remark}[Rappel : Mesure]
    Une fonction $ \mu : (X, \mathcal{B}) \longrightarrow \overline{\R_+}$ telle que : 
    \begin{enumerate}
        \item $ \mu(\emptyset) = 0 $ 
        \item \textbf{(Sigma-additivité)} : $ \forall (A_n)_{n \in \N}$ suite de parties mesures \textbf{deux à deux disjointes}, on ait : 
            \[ \mu \left( \bigcup_{n \in \N} A_n \right) = \sum_{n \in \N} \mu(A_n) \]
        est appelée mesure sur l'espace $(X, \mathcal{B}, \mu)$ alors appelé \textbf{espace mesuré}. 
    \end{enumerate}
\end{remark}

\subsection{Variable Aléatoire}

Pour pouvoir quantifier des calculs de probabilités ou ce que nous appellerons plus tard des lois, nous devons définir 
les variables aléatoires. 

\begin{definition}[Variable Aléatoire]
    Une variable aléatoire est une fonction mesurable qui associe une valeur numérique à chaque issue d'un espace probabilisé. 

    \vspace{0.5cm}

    Plus formellement, une variable aléatoire $X$ sur un espace probabilisé $(\Omega, \mathcal{F}, \myP)$ 
    est une fonction $ X : \Omega \longrightarrow \R$ telle que, pour tout ensemble $B \in \mathcal{B}_\R$ (tribu borélienne), 
    on ait $X^{-1} (B) \subset \mathcal{F}$. 

    \vspace{0.5cm}

    On définit l'ensemble des valeurs possibles de la variable aléatoire comme l'image de $\Omega$ par $X$ noté $X(\Omega)$. 
\end{definition}

La mesurabilité d'une variable aléatoire permet donc garantir que les évènements associés aux valeurs de la variable 
aléatoire sont bien mesurables par la mesure de probabilité. 

\begin{proposition}
    Maintetant que nous avons définit formellement le concept de variable aléatoire, on peut lier cette définition 
    à celle des évènements. En effet, une variable aléatoire est une fonction mesurable sur un espace probabilisé $(\Omega, \mathcal{F}, \myP)$
    telle que : 
        \[ X : \Omega \longrightarrow \R \quad \forall B \in \mathcal{B}_\R, X^{-1} (B) \subset \mathcal{F} \] 
    On peut alors caractériser un évènement $A$ comme la préimage d'un sous-ensemble de $\R$ par $X$ de la façon suivante. 
        \[ A = X^{-1}(B) \; \text{pour un certain} \; B \in \mathcal{B}_\R \] 
\end{proposition}


\subsection{Variables Discrètes et Continues}

Selon la nature de l'espérience aléatoire et de l'univers choisis, on distingue deux grands types de variables aléatoires : 
les variables aléatoires \textbf{discrètes} et \textbf{continues}. Cette distinction est fondamentale car elle 
caractérise la façon dont on exprime et calcule ensuite les probabilités.

\begin{definition}[Variable aléatoire discrète]
    Une variable aléatoire $X$ définie sur un espace probabilisé $(\Omega, \mathcal{F}, \myP)$ est dite 
    \textbf{discrète} si l'ensemble de ses valeurs possibles $X(\Omega)$ est un ensemble \textbf{fini ou dénombrable}. 
    Dans ce cas, la loi de probabilité de $X$ est donnée par une fonction de masse et les probabilités 
    s'expriment comme des sommes. 
\end{definition}

\begin{example}
    Le résultat d'un lancer de dé est une variable aléatoire discrète. 
    Par exemple, si $X$ désigne le résultat d’un lancer de dé, alors $X(\Omega) = \{1,2,3,4,5,6\}$.
\end{example}

\begin{definition}[Variable aléatoire continue]
    Une variable aléatoire $X$ est dite \textbf{continue} si elle peut prendre une infinité non dénombrable de valeurs, typiquement un intervalle de $\R$. 
    Dans ce cas, il n'existe pas de fonction de masse mais une \textbf{densité de probabilité}, et les probabilités s'expriment 
    par des \textbf{intégrales}.
\end{definition}

\begin{example}
    Le temps d'attente avant un événement (modélisé par une loi exponentielle), ou la taille d'une personne (loi normale), 
    sont des variables continues. Si $X$ est la taille d’un individu, alors $X(\Omega) \subseteq \R$ est un intervalle de réels.
\end{example}

\begin{remark}
    La distinction entre lois discrètes et lois continues repose donc sur la nature de la mesure de probabilité utilisée :
    \begin{itemize}
        \item \textbf{Mesure de comptage} (ou somme de Dirac) $\Rightarrow$ lois discrètes.
        \item \textbf{Mesure de Lebesgue} (avec densité) $\Rightarrow$ lois continues.
    \end{itemize}
\end{remark}

Nous allons maintenant étudier les deux grands types de lois de probabilité selon la nature de la variable aléatoire : 

\begin{itemize}
    \item Dans le cas discret : lois binomiale, géométrique, de Poisson...
    \item Dans le cas continu : loi uniforme, loi exponentielle, loi normale...
\end{itemize}